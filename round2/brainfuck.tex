\documentclass{../codeproblem}

\begin{document}

\title{Protecting Secrets}

\begin{flavor}
  The next order of business for the Resistance is to develop our own secret code for communication - after all, those are basically unbreakable, right?
\end{flavor}

\section*{Description}
Our super secret code, is defined as a simple series of commands that modify a list of 8 integers. Each of the 8 integers starts with a value of 0. Initially, the ``current integer'' is the first integer in the list.  Each of the commands are as follows:

\begin{center}
\begin{tabular}{| c | l |}\hline
  + & Increase the value of the current integer.\\\hline
  - & Decrease the value of the current integer.\\\hline
  > & Move to the next integer.\\\hline
  < & Move to the previous integer.\\\hline
  [ & If the current integer is 0, skip to after the matching ].\\\hline
  ] & If the current integer is not 0, skip back to the matching [.\\\hline
\end{tabular}
\end{center}

Consider the list to wrap fully. That is, if a > occurs while the current integer is the final integer, the new current integer will be the first integer. Integers should be positive and range from 0 to 255, wrapping appropriately.

The input will consist of a sequence of these commands no longer than 20 in length. Output will consist of the values of the 8 integers, space-separated.

You can assume the given code will eventually terminate.

\section*{Example}
\begin{example}
+++[>+>+>++>+++<<<<-]
|\textbf{0 3 3 6 9 0 0 0}||\end{example}

\end{document}
